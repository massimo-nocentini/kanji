
\documentclass[landscape]{article}

\usepackage[utf8]{inputenc}
\usepackage{LobsterTwo}
\usepackage[T1]{fontenc}
\usepackage{CJKutf8}
%\usepackage{fullpage} % 1" margins
\usepackage{graphicx}
\usepackage[a3paper]{geometry}
\usepackage{multicol}

\setlength{\columnsep}{1cm}


\begin{document}

\pagestyle{empty}

%\hspace{-4cm}

\huge
\LobsterTwo

\textbf{1Cor 13, 1-13}

\vfill

Se parlassi le lingue degli uomini e degli angeli, ma non avessi 
la carit\`a, sarei come bronzo che rimbomba o come cimbalo che strepita.

\vfill

E se avessi il dono della profezia, se conoscessi tutti i misteri e avessi
tutta la conoscenza, se possedessi tanta fede da trasportare le montagne,
ma non avessi la carit\`a, non sarei nulla.

\vfill

E se anche dessi in cibo tutti i miei beni e consegnassi il mio corpo
per averne vanto, ma non avessi la carit\`a, a nulla mi servirebbe.

\vfill

La carit\`a \`e magnanima, benevola \`e la carit\`a; non \`e invidiosa,
non si vanta, non si gonfia d'orgoglio, non manca di rispetto, non
cerca il proprio interesse, non si adira, non tiene conto del male 
ricevuto, non gode dell'ingiustizia ma si rallegra della verit\`a. 
Tutto scusa, tutto crede, tutto spera, tutto sopporta.

\vfill

La carit\`a non avr\`a mai fine. Le profezie scompariranno, il dono
delle lingue cesser\`a e la conoscenza svanir\`a. Infatti, in modo
imperfetto noi conosciamo e in modo imperfetto profetizziamo. Ma quando
verr\`a ci\`o che \`e perfetto, quello che \`e imperfetto scomparir\`a.
Quand'ero bambino, parlavo da bambino, pensavo da bambino, ragionavo da 
bambino. Divenuto uomo, ho eliminato ci\`o che \`e da bambino.

\vfill

Adesso noi vediamo in modo confuso, come in uno specchio; allora
invece vedremo faccia a faccia. Adesso conosco in modo imperfetto,
ma allora conoscer\`o perfettamente, come anch'io sono conosciuto.
Ora dunque rimangono queste tre cose: la fede, la speranza e la carit\`a.
Ma la pi\`u grande di tutte \`e la carit\`a!

\vfill

\end{document}
