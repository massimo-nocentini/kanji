
\documentclass[twocolumn]{article}

\usepackage[utf8]{inputenc}
\usepackage{LobsterTwo}
\usepackage[T1]{fontenc}
\usepackage{CJKutf8}
%\usepackage{fullpage} % 1" margins
\usepackage{graphicx}
\usepackage[a3paper]{geometry}
\usepackage{multicol}

\setlength{\columnsep}{1cm}

\title{Manifesto of a Doer}
\date{\vspace{-5ex}}


\begin{document}


\maketitle
\pagestyle{empty}
\pagenumbering{gobble}

\hspace{-4cm}

\Large
\LobsterTwo

\vfill

1, Leverage your energy. You can't increase the number of hours in a day, but you can multiply your effort. Understand the power of the influencers: The few influence the many. Find your multiplier. The person, the company, the organisations who can accelerate the change you want to make.

\vfill

2, Avoid easy deadlines. Deadlines serve you best when they are short, hard and, at first glance, impossible. Urgency gets things done.

\vfill

3, Follow through. On the big things. On the small things. Create a habit of always following through. As habits go, it's a good one to have.

\vfill

4, Focus on the task. If you are doing something that isn't pushing the task forward, that is called distraction. Distractions are plentiful. But remember, distractions stop you from doing.

\vfill

5, Obstacles will come your way. Guaranteed. Think of them as a gift. They will make you stronger. They will make you more creative. Rather than break you, they will define you.

\vfill

6, Ideas change things. But ideas by themselves change nothing. An idea needs effort to make it happen. Do the work.

\vfill

7, If you find something that you want to change, you have two options. One, is to talk about the change you are going to make. Or, two, stop talking. And begin.

\vfill

8, What you are doing is hard, but not impossible. Practice optimism.

\vfill

9, What is the priority today? Ask yourself this every day. It's your job to keep the main thing the main thing.

\vfill

10, The energy available to get this done is directly proportional to how much it matters to you. Only commit to things that matter.

\vfill

11, Perfection comes over time. Not at the beginning. Start where you are. But start.

\vfill

12, Sprint. Rest. Sprint. Rest. Human's get more done in bursts followed by rest. Getting things done isn't about who does the longest hours, but who does the smartest hours.

\vfill

13, 80\% of your time is spent on things that you are not good at. 20\% of your time is spent on the things you are very good at. In order to get more done, flip that.

\vfill

14, Teams multiply change. Teams with a clear purpose, and a clear sense of the change they can make, get the most done.

\vfill

15, Keep your energy for pushing forward. The past is done. Things out of your control cannot be changed. Energy spent being angry, jealous, or cynical is negative energy. Stay positive.

\vfill

16, Make a plan. Then accept it can and will change. Making something happen is about being nimble and adaptable.

\vfill

17, Say no. And say it often. As David Allen says: ``You can do anything, but not everything.'' Protect your time.

\vfill

18, Making things happen is fun. Making things happen that matter with a team as crazy as you are, is the best fun of all.

\vfill

19, Little actions repeated relentlessly result in big change. Don't underestimate the importance of \emph{small} multiplied by \emph{often}.

\vfill

20, Make a pact with failure early on. Respect it. But don't fear it. If it occupies your mind whilst doing, it can stop you from winning. Free your mind.

\vfill

21, Even though you are busy, make time to help others who are at the start of their journey. Give back. It will help you.

\vfill

22, All teams want to be part of history. Have something big that you want to change. This is bigger than you. Your purpose multiplies the team's stubbornness to get this thing done.

\vfill

23, If you are going to make change happen, make it a good one. This planet needs as many friends as it can get.

\vfill

\end{document}
